\documentclass[12pt]{scrartcl}

\usepackage[utf8]{inputenc}

\usepackage{mathpazo} % math & rm
% \linespread{1.05}        % Palatino needs more leading (space between lines)
\usepackage[scaled]{helvet} % ss
\usepackage{courier} % tt
\normalfont
\usepackage[french]{babel}
\usepackage[T1]{fontenc}

\usepackage{amsthm,amssymb,amsbsy,amsmath,amsfonts,amssymb,amscd}
\usepackage{dsfont}
\usepackage{tasks}
\usepackage{enumitem}
\usepackage[top=2cm, bottom=3cm, left=3cm , right=3cm]{geometry}
\usepackage{tikz}
\usetikzlibrary{automata,arrows,positioning,calc}
\newcommand{\R}{\texttt{R}}

\begin{document}
\begin{center}
	\rule{\textwidth}{1pt}
	\\ \ \\
	{\LARGE \textbf{TP cours 1 : Mise en place des TP}}\\
	\vspace{3mm}
	{\large Ecole-chercheur Mexico, La Rochelle \\ \vspace{3mm}}
	{\normalsize V. Picheny - N. Durrande}\\
	\vspace{3mm}
	\rule{\textwidth}{1pt}
	\vspace{5mm}
\end{center}
Cette session est dédiée à la mise en place du cas-test ``piton de la fournaise'' et à une première exploration à l'aide de différents plans d'expériences.

\section{Description et installation du cas-test}
Se référer au document \texttt{volcan\_test\_case.pdf}.

\section{Génération de différents plans d'expériences}

Les objectifs de cette section sont 1- d'observer différentes caractéristiques de plans d'expériences et 2- de générer un premier jeu de données pour le cas-test qui servira dans les TPs suivants.

\paragraph{Q1.} Ecrire une fonction qui implémente un plan d'expériences uniforme (à l'aide des fonctions \R \texttt{runif}, \texttt{matrix}). 
Cette fonction doit prendre en paramètres le nombre de points $n$ et le nombre de variables $d$, et retourner une matrice $n \times d$ de valeurs entre 0 et 1. Générer un plan à 100 points en dimension 5. Visualiser.

\paragraph{Q2.} Ecrire une fonction qui implémente un hypercube latin (aléatoire) : on pourra utiliser la fonction \texttt{sample}. Cette fonction utilisera les mêmes entrées / sorties que la précédente.
Attention à la normalisation. Cette question peut être sautée par les stagiaires souhaitant directement explorer le cas-test. Générer un plan à 100 points en dimension 5. Visualiser.

\paragraph{Q3.} A l'aide du paquet \texttt{DiceDesign}, générer un hypercube latin optimisé à 100 points et 5 variables, par exemple avec la fonction \texttt{maximinESE\_LHS}, ou bien avec le paquet \texttt{lhs} et la fonction \texttt{improvedLHS}.

\paragraph{Q4.} A l'aide du paquet \texttt{DiceDesign}, générer 100 points d'une suite à faible discrépance en dimension $d=5$ à l'aide de \texttt{runif.faure}.

\paragraph{Q5.} Comparer les 4 plans d'expériences générés. 
On pourra regarder les répartitions sur les marginales d'ordre 1 à l'aide d'histogrammes (\texttt{hist}) ou de graphes en bâton (\texttt{plot(seqce,rep(1,N),type="h")}), 
les répartitions sur les espaces bi-dimensionnels (fonction \texttt{pairs}) et encore des métriques de remplissage d'espace (\texttt{minDist}, \texttt{meshRatio} de \texttt{DiceDesign}).

\paragraph{Q6.} Choisir un des plans et effectuer les appels au modèle de volcan correspondant. A noter : on a uniquement besoin d'utiliser la fonction \texttt{compute\_wls}, qui accepte en entrée un vecteur ou une matrice.

\paragraph{Q7.} Etudier les données générées. On pourra regarder en particulier comment la sortie du modèle varie en fonction de chaque paramètre d'entrée. 
Pour un plan d'expériences \texttt{X} et un vecteur de réponse \texttt{Y} : \texttt{plot(X[,1], Y}, etc.

\end{document}