\documentclass{beamer}

\usepackage[utf8]{inputenc}
\usepackage[french]{babel}
\usepackage[T1]{fontenc}
\usepackage{verbatim}
\usepackage{graphicx}
\usepackage{amsthm,amssymb,amsbsy,amsmath,amsfonts,amssymb,amscd}
\usepackage{dsfont}
\usepackage{array}
\usepackage{fancyvrb}

\title{TP du cours métamodélisation - corrigé}
\author{Nicolas Durrande, Victor Picheny,}
\date{\null}

%%%%%%%%%%%%%%%%%%%%%%%%%%%%%%%%%%%%%%%%%%%%%%%%%%%%%%
%%%%%%%%%%%%%%%%%%%%%%%%%%%%%%%%%%%%%%%%%%%%%%%%%%%%%%
%%%%%%%%%%%%%%%%%%%%%%%%%%%%%%%%%%%%%%%%%%%%%%%%%%%%%%
\begin{document}

%%%%%%%%%%%%%%%%%%%%%%%%%%%%%%%%%%%%%%%%%%%%%%%%%%%%%%
\begin{frame}
  \titlepage
\end{frame}

%%%%%%%%%%%%%%%%%%%%%%%%%%%%%%%%%%%%%%%%%%%%%%%%%%%%%%
\begin{frame}[fragile]{Q1}
\begin{verbatim}
 plot(X, Y)
\end{verbatim}
\includegraphics[trim=0mm 0mm 0mm 0mm, clip, width=\textwidth]{plotXY.pdf}
La fonction semble très régulière, probablement croissante.
\end{frame}

%%%%%%%%%%%%%%%%%%%%%%%%%%%%%%%%%%%%%%%%%%%%%%%%%%%%%%
\begin{frame}[fragile]{Q2 et Q3}
 \begin{Verbatim}[fontsize=\small]
m <- km(y~1, design=data.frame(x=X), response=data.frame(y=Y),
        coef.trend = 0, nugget=1e-8,
        covtype="gauss", coef.var=4, coef.cov=.5)

x <- matrix(seq(-0.2, 1.2, 0.01))
Z <- t(simulate(m, 10, newdata=data.frame(x=x), cond=FALSE))
matplot(x, Z, type='l', col=1)

Zc <- t(simulate(m, 10, newdata=data.frame(x=x), cond=TRUE))
matplot(x, Zc, type='l', col=1)
points(X, Y)
\end{Verbatim}

\texttt{km} crée un modèle de krigeage, et \texttt{simulate} génère des trajectoires correspondant aux points \texttt{x}. 
En changeant \texttt{cond=TRUE}, on conditionne aux observations (les trajectoires passent par les observations).

\end{frame}
%%%%%%%%%%%%%%%%%%%%%%%%%%%%%%%%%%%%%%%%%%%%%%%%%%%%%%
\begin{frame}{Trajectoires interpolantes ou non, mais de même régularité}
\includegraphics[trim=0mm 0mm 0mm 0mm, clip, width=.7\textwidth]{traj1.pdf} 
\includegraphics[trim=0mm 0mm 0mm 0mm, clip, width=.7\textwidth]{traj2.pdf} 
\end{frame}
%%%%%%%%%%%%%%%%%%%%%%%%%%%%%%%%%%%%%%%%%%%%%%%%%%%%%%
\begin{frame}[fragile]{Q4: étude d'un modèle \texttt{km}}
 \begin{Verbatim}[fontsize=\tiny]
m <- km(y~1, design=data.frame(x=X), response=data.frame(y=Y), 
        covtype="gauss", coef.trend = 0, nugget=1e-8)
print(m)

Trend  coeff.:
                       
 (Intercept)     0.0000

Covar. type  : gauss 
Covar. coeff.:
               Estimate
    theta(x)     0.5349

Variance estimate: 35.01083

Nugget effect : 1e-08
\end{Verbatim}
\begin{itemize}
 \item \texttt{Trend coeff} : moyenne, ici imposée égale à zéro
 \item \texttt{Covar. coeff} : paramètre de portée (homogène à une distance), estimé par max de vraisemblance
 \item \texttt{Variance estimate} : paramètre de variance du processus, estimé par max de vraisemblance
\end{itemize}

\end{frame}
%%%%%%%%%%%%%%%%%%%%%%%%%%%%%%%%%%%%%%%%%%%%%%%%%%%%%%
\begin{frame}[fragile]{Q5}
 \begin{verbatim}
res <-leaveOneOut.km(m, type='SK')
res_std <- (Y-res$mean) / res$sd

hist(res_std, freq=FALSE, xlim=c(min(X, -3), max(X, 3)))
x <- seq(-3, 3, 0.03)
lines(x, dnorm(x))
\end{verbatim}
\includegraphics[trim=0mm 0mm 0mm 0mm, clip, width=.7\textwidth]{XVerror.pdf} 

Ici : les erreurs par validation croisée, une fois normalisées, sont raisonnablement gaussiennes centrées réduites.
\end{frame}
%%%%%%%%%%%%%%%%%%%%%%%%%%%%%%%%%%%%%%%%%%%%%%%%%%%%%%
\begin{frame}[fragile]{Q5 : \texttt{plot(m)}}
\includegraphics[trim=0mm 0mm 0mm 0mm, clip, width=.6\textwidth]{plotm.pdf} 
\begin{itemize}
 \item Premier graphe : les erreurs sont toutes très faibles (modèle très précis)
 \item Deuxième graphe : on ne voit pas de structure pour les erreurs (OK)
 \item Troisième graphe : les résidus sont bien gaussiens (points sur la droite des quantiles)
\end{itemize}

\end{frame}
%%%%%%%%%%%%%%%%%%%%%%%%%%%%%%%%%%%%%%%%%%%%%%%%%%%%%%
\begin{frame}[fragile]{Q6}
 \begin{Verbatim}[fontsize=\small]
sectionview(m, xlim=c(-0.2, 4), ylim=c(-18, 22))

# modèle de krigeage ordinaire (estimation tendance constante)
m <- km(y~1, design=data.frame(x=X), response=data.frame(y=Y), 
        covtype="gauss", nugget=1e-8)
sectionview(m, xlim=c(-0.2, 4), ylim=c(-18, 22))

# modèle de krigeage universel (estimation tendance lineaire)
m <- km(y~x, design=data.frame(x=X), response=data.frame(y=Y), 
        covtype="gauss", nugget=1e-8)
sectionview(m, xlim=c(-0.2, 4), ylim=c(-18, 22))
\end{Verbatim}
\end{frame}
%%%%%%%%%%%%%%%%%%%%%%%%%%%%%%%%%%%%%%%%%%%%%%%%%%%%%%
\begin{frame}[fragile]{Q6}
\begin{columns}
 \begin{column}{.49\textwidth}
  \includegraphics[trim=0mm 0mm 0mm 0mm, clip, width=\textwidth]{GP2.pdf}
\includegraphics[trim=0mm 0mm 0mm 0mm, clip, width=\textwidth]{GP3.pdf} 
\includegraphics[trim=0mm 0mm 0mm 0mm, clip, width=\textwidth]{GP4.pdf} 
 \end{column}
 \begin{column}{.49\textwidth}
 En l'absence d'observations proches : le krigeage effectue un ``retour à la moyenne''.
 \smallskip
 
  Choisir la tendance permet d'extrapoler... mais peut aussi amplifier l'erreur !

 \end{column}
\end{columns}
\end{frame}
%%%%%%%%%%%%%%%%%%%%%%%%%%%%%%%%%%%%%%%%%%%%%%%%%%%%%%%%%%%%%%%%%%%%%%%%%%%%%%%
\begin{frame}[fragile]{Q7 et Q8}
 \begin{Verbatim}[fontsize=\tiny]
km(formula = y ~ ., design = data.frame(x = X), response = data.frame(y = Y), 
    covtype = "matern5_2", nugget = 1e-08)

Trend  coeff.:
               Estimate
 (Intercept)    -1.5942
         x.1     0.0819
         x.2    -0.0102
         x.3     0.8546
         x.4     2.5819
         x.5    -0.1406

Covar. type  : matern5_2 
Covar. coeff.:
               Estimate
  theta(x.1)     1.9621
  theta(x.2)     1.9661
  theta(x.3)     1.6171
  theta(x.4)     0.4056
  theta(x.5)     0.0561

Variance estimate: 0.2128299
\end{Verbatim}
On observe bien : un paramètre de tendance et de portée par $x.i$. Sachant que chaque $x.i$ varie entre 0 et 1, une portée égale à 2 est très élevée (le krigeage est presque constant dans cette direction, $x.i$ est peu influent), 
et de 0.05 très petite (beaucoup de variation dans cette direction).
\end{frame}
%%%%%%%%%%%%%%%%%%%%%%%%%%%%%%%%%%%%%%%%%%%%%%%%%%%%%%
\begin{frame}[fragile]{Q8}
\includegraphics[trim=0mm 0mm 0mm 0mm, clip, width=.8\textwidth]{plotvolcan.pdf} 

Validation : erreur par validation croisée plutôt bonne, quelques points imprécis pour les valeurs faibles de $Y$. QQ-plot : OK sauf pour les quantiles faibles.

\end{frame}
%%%%%%%%%%%%%%%%%%%%%%%%%%%%%%%%%%%%%%%%%%%%%%%%%%%%%%
\begin{frame}[fragile]{Q8}
\includegraphics[trim=0mm 0mm 0mm 0mm, clip, width=.8\textwidth]{GPvolcan.pdf} 

On retrouve bien : un modèle très ``plat'' pour $x.1$, $x.2$, $x.3$, très variable pour $x.5$.
\end{frame}
%%%%%%%%%%%%%%%%%%%%%%%%%%%%%%%%%%%%%%%%%%%%%%%%%%%%%%
\begin{frame}[fragile]{Q10}
 \begin{Verbatim}[fontsize=\tiny]
getmean <- function(newdata, m) {
  pred <- predict(object=m, newdata=newdata, type="UK")
  return(pred$mean)
}
X1 <- data.frame(randomLHS(10000,5))
X2 <- data.frame(randomLHS(10000,5))
colnames(X1) <- colnames(X2) <- colnames(m@X)
res2 <- soboljansen(model = getmean, X1=X1, X2=X2, nboot = 50, conf = 0.95, m=m)
\end{Verbatim}
\end{frame}
%%%%%%%%%%%%%%%%%%%%%%%%%%%%%%%%%%%%%%%%%%%%%%%%%%%%%%
\begin{frame}[fragile]{Q10}
\includegraphics[trim=0mm 0mm 0mm 0mm, clip, width=.8\textwidth]{sobolGPm.pdf} 
Très rapide... mais ne prend pas en compte l'erreur supplémentaire due au métamodèle.
\end{frame}
%%%%%%%%%%%%%%%%%%%%%%%%%%%%%%%%%%%%%%%%%%%%%%%%%%%%%%
\begin{frame}[fragile]{Q10}
 \begin{Verbatim}[fontsize=\tiny]
X1 <- data.frame(randomLHS(1000,5))
X2 <- data.frame(randomLHS(1000,5))
candidate <- data.frame(randomLHS(100,5))
colnames(X1) <- colnames(X2) <- colnames(candidate) <- colnames(m@X)
res <- sobolGP(model = m, type="UK", MCmethod="soboljansen",
               X1=X1, X2=X2, nsim = 20, nboot=50, sequential = TRUE, candidate=candidate)
plot(res)
\end{Verbatim}
On est forcé de réduire la taille de X1 et X2.
\end{frame}
%%%%%%%%%%%%%%%%%%%%%%%%%%%%%%%%%%%%%%%%%%%%%%%%%%%%%%
\begin{frame}[fragile]{Q10}
\includegraphics[trim=0mm 0mm 0mm 0mm, clip, width=.5\textwidth]{sobolGP.pdf} 
Beaucoup plus long ! Mais on peut attribuer la part d'erreur due à Monte-Carlo (la taille de X1 et X2) et celle due au krigeage (essentiellement due à la taille du plan d'expériences).
Ici : le modèle est très précis, donc l'erreur vient essentiellement du MC.
\end{frame}
%%%%%%%%%%%%%%%%%%%%%%%%%%%%%%%%%%%%%%%%%%%%%%%%%%%%%%
\end{document}