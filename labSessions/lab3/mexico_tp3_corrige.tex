\documentclass{beamer}

\usepackage[utf8]{inputenc}
\usepackage[french]{babel}
\usepackage[T1]{fontenc}
\usepackage{verbatim}
\usepackage{graphicx}
\usepackage{amsthm,amssymb,amsbsy,amsmath,amsfonts,amssymb,amscd}
\usepackage{dsfont}
\usepackage{array}
\usepackage{fancyvrb}

\title{TP du cours optim avec métamodélisation - corrigé}
\author{Nicolas Durrande, Victor Picheny,}
\date{\null}

%%%%%%%%%%%%%%%%%%%%%%%%%%%%%%%%%%%%%%%%%%%%%%%%%%%%%%
%%%%%%%%%%%%%%%%%%%%%%%%%%%%%%%%%%%%%%%%%%%%%%%%%%%%%%
%%%%%%%%%%%%%%%%%%%%%%%%%%%%%%%%%%%%%%%%%%%%%%%%%%%%%%
\begin{document}

%%%%%%%%%%%%%%%%%%%%%%%%%%%%%%%%%%%%%%%%%%%%%%%%%%%%%%
\begin{frame}
  \titlepage
\end{frame}
%%%%%%%%%%%%%%%%%%%%%%%%%%%%%%%%%%%%%%%%%%%%%%%%%%%%%%
\begin{frame}[fragile]{LHS initial}
\includegraphics[trim=0mm 0mm 0mm 0mm, clip, width=\textwidth]{XY.pdf} 

Pas grand chose à voir... à part la 4e variable.

\end{frame}
%%%%%%%%%%%%%%%%%%%%%%%%%%%%%%%%%%%%%%%%%%%%%%%%%%%%%%
%%%%%%%%%%%%%%%%%%%%%%%%%%%%%%%%%%%%%%%%%%%%%%%%%%%%%%%%%%%%%%%%%%%%%%%%%%%%%%%
\begin{frame}[fragile]{Modèle initial}
 \begin{Verbatim}[fontsize=\tiny]
km(formula = y ~ ., design = data.frame(x = X), response = data.frame(y = Y), 
    covtype = "matern5_2", nugget = 1e-08)

Trend  coeff.:
               Estimate
 (Intercept)    -1.5942
         x.1     0.0819
         x.2    -0.0102
         x.3     0.8546
         x.4     2.5819
         x.5    -0.1406

Covar. type  : matern5_2 
Covar. coeff.:
               Estimate
  theta(x.1)     1.9621
  theta(x.2)     1.9661
  theta(x.3)     1.6171
  theta(x.4)     0.4056
  theta(x.5)     0.0561

Variance estimate: 0.2128299
\end{Verbatim}
On observe bien : un paramètre de tendance et de portée par $x.i$. Sachant que chaque $x.i$ varie entre 0 et 1, une portée égale à 2 est très élevée (le krigeage est presque constant dans cette direction, $x.i$ est peu influent), 
et de 0.05 très petite (beaucoup de variation dans cette direction).
\end{frame}
%%%%%%%%%%%%%%%%%%%%%%%%%%%%%%%%%%%%%%%%%%%%%%%%%%%%%%
\begin{frame}[fragile]
\includegraphics[trim=0mm 0mm 0mm 0mm, clip, width=.8\textwidth]{plotvolcan.pdf} 

Validation : erreur par validation croisée plutôt bonne, quelques points imprécis pour les valeurs faibles de $Y$. QQ-plot : OK sauf pour les quantiles faibles.

\end{frame}
%%%%%%%%%%%%%%%%%%%%%%%%%%%%%%%%%%%%%%%%%%%%%%%%%%%%%%
\begin{frame}[fragile]
\includegraphics[trim=0mm 0mm 0mm 0mm, clip, width=.8\textwidth]{GPvolcan.pdf} 

On retrouve bien : un modèle très ``plat'' pour $x.1$, $x.2$, $x.3$, très variable pour $x.5$.
\end{frame}
%%%%%%%%%%%%%%%%%%%%%%%%%%%%%%%%%%%%%%%%%%%%%%%%%%%%%%
\begin{frame}[fragile]{Tendance, noyaux}
\begin{Verbatim}[fontsize=\scriptsize]
m1 <- km(y~1, design=data.frame(x=X), response=data.frame(y=Y), 
        covtype="gauss", nugget=1e-8)

m2 <- km(y ~ .^2 + xs^2 + ys^2 + zs^2 + a^2 + p^2, design=data.frame(X), 
         response=data.frame(y=Y), 
         covtype="gauss", nugget=1e-8, lower=rep(0.05, 5))

m3 <- km(y~., design=data.frame(x=X), response=data.frame(y=Y), 
         covtype="matern5_2", nugget=1e-8)

c(logLik(m1), logLik(m2), logLik(m3))
 -117.294340    2.902064   -2.791663

c(Q2m1, Q2m2, Q2m3)
 0.9576836 0.9716956 0.9662655 
\end{Verbatim} 
 
\end{frame}      
%%%%%%%%%%%%%%%%%%%%%%%%%%%%%%%%%%%%%%%%%%%%%%%%%%%%%%
\begin{frame}[fragile]{Tendance, noyaux}

\begin{itemize}
 \item pas de choix de noyau ou tendance évident...
 \item la solution optimale pour la vraisemblance ou le Q2 dépend de votre plan initial !
 \item optimisation de la vraisemblance locale ! sauf si \texttt{optim.method=``gen''}
\end{itemize}

\begin{alertblock}{En fait : fonction très piégeuse, difficile de trouver un noyau qui convient dans toutes les directions}
 \includegraphics[trim=0mm 0mm 0mm 0mm, clip, width=\textwidth]{cut1.pdf} 
\end{alertblock}
\end{frame}
%%%%%%%%%%%%%%%%%%%%%%%%%%%%%%%%%%%%%%%%%%%%%%%%%%%%%%
\begin{frame}[fragile]{EGO}
\begin{Verbatim}
res <- EGO.nsteps(model=m, fun=compute_wls, nsteps=50, 
                  lower=rep(0, 5), upper=rep(1,5))
                  
names(res)
 "par"       "value"     "npoints"   "nsteps"    "lastmodel"
\end{Verbatim} 
\end{frame}
%%%%%%%%%%%%%%%%%%%%%%%%%%%%%%%%%%%%%%%%%%%%%%%%%%%%%%
\begin{frame}[fragile]{EGO}
\includegraphics[trim=0mm 0mm 0mm 0mm, clip, width=\textwidth]{evolEGO.pdf} 
\end{frame}
%%%%%%%%%%%%%%%%%%%%%%%%%%%%%%%%%%%%%%%%%%%%%%%%%%%%%%
\begin{frame}[fragile]{}
\includegraphics[trim=0mm 0mm 0mm 0mm, clip, width=\textwidth]{pairsEGO.pdf} 
\end{frame}
%%%%%%%%%%%%%%%%%%%%%%%%%%%%%%%%%%%%%%%%%%%%%%%%%%%%%%
\begin{frame}[fragile]{}
\includegraphics[trim=0mm 0mm 0mm 0mm, clip, width=\textwidth]{sectionEGO.pdf} 
\end{frame}

\end{document}

%%%%%%%%%%%%%%%%%%%%%%%%%%%%%%%%%%%%%%%%%%%%%%%%%%%%%%
\begin{frame}[fragile]{Q10 : analyse de sensibilité}
 \begin{Verbatim}[fontsize=\tiny]
getmean <- function(newdata, m) {
  pred <- predict(object=m, newdata=newdata, type="UK")
  return(pred$mean)
}
X1 <- data.frame(randomLHS(10000,5))
X2 <- data.frame(randomLHS(10000,5))
colnames(X1) <- colnames(X2) <- colnames(m@X)
res2 <- soboljansen(model = getmean, X1=X1, X2=X2, nboot = 50, conf = 0.95, m=m)
\end{Verbatim}
\end{frame}
%%%%%%%%%%%%%%%%%%%%%%%%%%%%%%%%%%%%%%%%%%%%%%%%%%%%%%
\begin{frame}[fragile]{Q10}
\includegraphics[trim=0mm 0mm 0mm 0mm, clip, width=.8\textwidth]{sobolGPm.pdf} 
Très rapide... mais ne prend pas en compte l'erreur supplémentaire due au métamodèle.
\end{frame}
%%%%%%%%%%%%%%%%%%%%%%%%%%%%%%%%%%%%%%%%%%%%%%%%%%%%%%
\begin{frame}[fragile]{Q10}
 \begin{Verbatim}[fontsize=\tiny]
X1 <- data.frame(randomLHS(1000,5))
X2 <- data.frame(randomLHS(1000,5))
candidate <- data.frame(randomLHS(100,5))
colnames(X1) <- colnames(X2) <- colnames(candidate) <- colnames(m@X)
res <- sobolGP(model = m, type="UK", MCmethod="soboljansen",
               X1=X1, X2=X2, nsim = 20, nboot=50, sequential = TRUE, candidate=candidate)
plot(res)
\end{Verbatim}
On est forcé de réduire la taille de X1 et X2.
\end{frame}
%%%%%%%%%%%%%%%%%%%%%%%%%%%%%%%%%%%%%%%%%%%%%%%%%%%%%%
\begin{frame}[fragile]{Q10}
\includegraphics[trim=0mm 0mm 0mm 0mm, clip, width=.5\textwidth]{sobolGP.pdf} 
Beaucoup plus long ! Mais on peut attribuer la part d'erreur due à Monte-Carlo (la taille de X1 et X2) et celle due au krigeage (essentiellement due à la taille du plan d'expériences).
Ici : le modèle est très précis, donc l'erreur vient essentiellement du MC.
\end{frame}
%%%%%%%%%%%%%%%%%%%%%%%%%%%%%%%%%%%%%%%%%%%%%%%%%%%%%%
\end{document}