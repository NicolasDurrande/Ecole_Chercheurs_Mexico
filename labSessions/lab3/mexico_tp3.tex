\documentclass[12pt]{scrartcl}

\usepackage[utf8]{inputenc}

\usepackage{mathpazo} % math & rm
% \linespread{1.05}        % Palatino needs more leading (space between lines)
\usepackage[scaled]{helvet} % ss
\usepackage{courier} % tt
\normalfont
\usepackage[french]{babel}
\usepackage[T1]{fontenc}

\usepackage{amsthm,amssymb,amsbsy,amsmath,amsfonts,amssymb,amscd}
\usepackage{dsfont}
\usepackage{tasks}
\usepackage{enumitem}
\usepackage[top=2cm, bottom=3cm, left=3cm , right=3cm]{geometry}
\usepackage{tikz}
\usetikzlibrary{automata,arrows,positioning,calc}
\newcommand{\R}{\texttt{R}}

\begin{document}
\begin{center}
	\rule{\textwidth}{1pt}
	\\ \ \\
	{\LARGE \textbf{TP cours 3 : Optimisation sur base de krigeage}}\\
	\vspace{3mm}
	{\large Ecole-chercheur Mexico, La Rochelle \\ \vspace{3mm}}
	{\normalsize N. Durrande - V. Picheny}\\
	\vspace{3mm}
	\rule{\textwidth}{1pt}
	\vspace{5mm}
\end{center}
\paragraph{}
L'objectif du TP consiste à trouver les paramètres du simulateurs numérique du volcan qui minimisent l'erreur de prédiction par rapport aux données observées par satellite. On s'intéresse donc à un problème de calibration que l'on va traiter comme un problème d'optimisation.

\paragraph{}
Ce TP comprend deux parties: pour commencer il faudra obtenir un bon modèle de krigeage qui approche la fonction \texttt{compute\_wls} (deuxième partie du précédent TP - qu'on approfondira ici). 
La seconde partie consiste à utiliser l'algorithme EGO du package \texttt{DiceOptim} pour trouver les paramètres qui minimisent cette erreur.

\paragraph{}
Le script \texttt{lab3\_script.R} vous est fourni pour vous faire gagner un peu de temps pour certaines questions... à vous de le compléter !
%%%%%%%%%%%%%%%%%%%%%%%%%%%%%%%%
\section{Modélisation par krigeage}

\paragraph{Q1.} Récupérer le meilleur plan d'expériences de 100 points en dimension 5 obtenu lors de la séance de lundi. 
Si vous n'êtes pas arrivé à un résultat convainquant vous pouvez utiliser le plan d'expériences et les observations fournies dans le fichier \texttt{XY\_volcano.Rdata} (à charger avec la fonction \texttt{load}).

\paragraph{Q2.} Tester différentes covariances et tendances en estimant les paramètres par maximum de vraisemblance. Obtenez-vous les mêmes résultats si vous executez plusieurs fois de suite la fonction \texttt{km} ? La fonction \texttt{logLik} est utile pour calculer la vraisemblance d'un modèle, ce qui permet de les comparer entre eux.

\paragraph{Q3. } Les valeurs des paramètres estimés peuvent-elle nous renseigner sur la fonction que l'on approche ? Proposer une interprétation. 

\paragraph{Q4. } Tester la qualité de prédiction de la moyenne de krigeage de différents modèles en calculant le critère 
$$Q_2 = 1 - \frac{\sum_{i=1}^n (y_i - m(x_i))^2}{\sum_{i=1}^n (y_i - mean(y_i)^2}$$
sur des résidus obtenus par leave-one-out (cf fonction \texttt{leaveOneOut.km}). Vous avez peut-être constaté précédement qu'ajouter des termes de tendance augmente toujours la vraisemblance du modèle mais est-ce que cela augmente forcément le $Q_2$ ? 

\paragraph{Q5. } Normaliser les résidus obtenus par leave-one-out et utiliser le code qui vous est fourni pour comparer leur distribution à une loi $\mathcal{N}(0, 1)$. 
Le résultat vous parrait-il satisfaisant ? Pour compléter, utiliser la fonction \texttt{plot(model)}. Choisir le modèle qui vous parrait le meilleur, c'est celui-là que l'on utilisera par la suite pour l'optimisation.

%%%%%%%%%%%%%%%%%%%%%%%%%%%%%%%%
\section{Optimisation Globale avec EGO}

\paragraph{Q6. } Charger la package \texttt{DiceOptim} et utiliser la fonction \texttt{EGO.nsteps} pour effectuer \texttt{nsteps=20} itérations de l'algorithme EGO. 
Cette fonction prend en entrée le modèle de krigeage ainsi que la fonction à optimiser \texttt{compute\_wls}. Par la suite, on notera \texttt{res} l'objet qui est retourné par la fonction \texttt{EGO.nsteps}.

\paragraph{Q7. } Observer les différents éléments contenus dans \texttt{res}. Utiliser le script fourni pour tracer l'évolution des valeurs de la fonction objectif aux points visités par EGO en fonction de l'itération. 
Quelle est la plus petite valeur observée et quels sont les paramètres associés ? Y a-t-il une amélioration par rapport à la meilleure valeur observée sur le plan d'expérience ?
Comment observe-t-on le compromis exploration / intensification ?

\paragraph{Q8. } Le code fourni permet de représenter la distribution dans l'espace des paramètres testés par l'optimiseur. Identifier des zones d'exploration et d'intensification.

\paragraph{Q9. } Utiliser la fonction \texttt{sectionview} de \texttt{DiceView} pour représenter une vue en coupe du modèle (\texttt{res\$lastmodel}) centrée sur l'optimum.

\paragraph{Q10. } Reprendre les questions 5 à 9 en augmentant la valeur de \texttt{nsteps}. De manière alternative, on peut essayer de relancer \texttt{EGO.nsteps} à partir du dernier modèle (\texttt{res\$lastmodel}), 
où d'un nouveau modèle bâti à partir du plan d'expériences enrichi des 20 points.

\paragraph{Q11 (bonus). } Reprendre les dernières questions sur l'analyse de sensibilité à l'aide de krigeage du TP précédent. Les résultats sont-il cohérents avec le comportement observé de l'algorithme EGO ?

\textit{Il y a 2 solutions : soit faire l'AS sur la moyenne du krigeage, soit faire l'AS sur des trajectoires.
Dans le premier cas, on a besoin d'encapsuler la sortie de la fonction \texttt{predict(m)}. Dans le deuxième cas, on peut directement utiliser la fonction \texttt{sobolGP} du package \texttt{sensitivity}.
Se reporter au script d'aide. Comparer les deux résultats obtenus et les temps de calcul respectifs.} 

%\paragraph{Q8 (bonus). } Comparer les résultats obtenus avec une optimisation par descente de gradient avec 120 évaluations de la fonction (et donc 20 itérations puisque, en dimension 5, chaque itération effectue 6 évaluation de la fonction pour évaluer la fonction et son gradient).
\end{document}
