\documentclass[12pt]{scrartcl}

\usepackage[utf8]{inputenc}

\usepackage{mathpazo} % math & rm
% \linespread{1.05}        % Palatino needs more leading (space between lines)
\usepackage[scaled]{helvet} % ss
\usepackage{courier} % tt
\normalfont
\usepackage[T1]{fontenc}

\usepackage{amsthm,amssymb,amsbsy,amsmath,amsfonts,amssymb,amscd}
\usepackage{dsfont}
\usepackage{tasks}
\usepackage{enumitem}
\usepackage[top=2cm, bottom=3cm, left=3cm , right=3cm]{geometry}
\usepackage{tikz}
\usepackage[hidelinks]{hyperref}
\usepackage{ulem}

\usetikzlibrary{automata,arrows,positioning,calc}


\begin{document}
\begin{center}
	\rule{\textwidth}{1pt}
	\\ \ \\
	{\LARGE \textbf{Lab 3 -- Gaussian Process Regression}}\\
	\vspace{3mm}
	{\large Short course on Statistical modelling for optimization\\ \vspace{3mm}}
	{\normalsize N. Durrande - J.C. Croix, Universidad Tecnol\'ogica de Pereira, 2017}\\
	\vspace{3mm}
	\rule{\textwidth}{1pt}
	\vspace{5mm}
\end{center}
The aim of this lab session is to obtain the best possible GPR model for the data that has been obtained with the catapult simulator.

%%%%%%%%%%%%%%%%%%%%%%%%%%%%%%%%%%%%%%%%%%%%%%%%%
\subsection*{GPR with GPy}
GPy is a python package for Gaussian process models. If you have not already installed it on your computer, we advise that you download the developers version on github \url{https://github.com/SheffieldML/GPy/tree/devel} and follow the instructions.% (there is a link ``Download ZIP'' on the right). The installation steps are: 1. unzip the file; 2. Open a terminal (for example the Anaconda terminal) and go to the unzipped folder; 3. Run the command \texttt{python setup.py install}. You should then be able to import the GPy library.

% \subsection*{Questions}
\paragraph{Q1.} Import the data (design of experiments and outputs from the simulator) you have generated.

\paragraph{Q2.}
The code for creating and optimizing a basic GP model is already given in the python script. Read it carefully to understand each line signification.

\paragraph{Q3.}
Compute the IMSE associated to the model based on a test set of 1000 points. What do you think about the model accuracy?

\paragraph{Q4.}
Plot the leave-one-out predictions against the actual observations. What do you think about the model accuracy?

\paragraph{Q5.}
Implement a function that returns the $Q^2$ criterion and compute its value on the LOO predictions. What component of the model are we testing here?

\paragraph{Q6.}
Compute the standardized LOO residuals and that compare them to the $\mathcal{N}(0,1)$ distribution. What does this test allows us to asses?

\paragraph{Q7.} Try various models and select the best one. When building the models, you may consider changing:
\begin{itemize}
   	\item the kernel (try various ones and sums of kernels)
   	\item the way kernel parameters are estimated (optimization staring point, boundaries, ...)
   	\item possible rotations of the input space
\end{itemize}

\end{document}
