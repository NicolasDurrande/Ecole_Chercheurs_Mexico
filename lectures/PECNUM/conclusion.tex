\begin{frame}
\frametitle{Conclusion}
\begin{block}{Pourquoi utiliser une stratégie séquentielle ?}
\begin{itemize}
 \item Flexibilité / contraintes imposées par l'étude
 \item Ajout possible d'objectifs guidés par les observations effectuées
\end{itemize}
\end{block}

\begin{exampleblock}{Stratégie séquentielle et métamodèle}
\begin{itemize}
 \item On exploite l'information donnée par le métamodèle pour choisir les observations
 \item L'information utile s'adapte à l'objectif poursuivi !
\end{itemize}
\end{exampleblock}

\begin{alertblock}{Augmentation de la complexité !}
\begin{itemize}
 \item Boucles d'optimisation imbriquées
 \item Mise à jour de modèles, etc.
\end{itemize}
\end{alertblock}

\end{frame}
%----------------------------------------
\begin{frame}
\frametitle{Généralisation de l'approche séquentielle}
\begin{block}{Un schéma unique}
\begin{itemize}
 \item Définition d'un critère (MSE, IMSE, $IMSE_T$, $EI$)
 \item Recherche du point optimal au sens de ce critère
 \item Enrichissement du métamodèle
\end{itemize}
\end{block}

\begin{exampleblock}{Stratégie séquentielle ``sur mesure''}
\begin{itemize}
 \item Il suffit de définir un critère correspondant au besoin !
 \item Beaucoup de travaux existants...
 \item ... un plus grand nombre encore de besoins spécifiques 
\end{itemize}
\end{exampleblock}
\end{frame}
