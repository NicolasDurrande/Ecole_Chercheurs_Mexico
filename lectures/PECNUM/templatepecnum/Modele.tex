\documentclass[10pt]{beamer}
\usetheme{Pecnum}
\usepackage{amsmath,amssymb}
\usepackage{colortbl}
%\usepackage{ucs}
\usepackage[utf8]{inputenc}
\usepackage{textcomp}
\usepackage{amsmath}
\usepackage{amsthm}
\usepackage{amsfonts}
\usepackage{amssymb}
\usepackage{amsbsy}
\setbeamercovered{dynamic}

\title{Titre de la présentation}
\author{Prénom Nom}
\date{\'Ecole Centrale de Lyon, 18 mai 2015}
\institute{\includegraphics[scale=.15]{logo.png}}

\begin{document}
\begin{frame}
    	\maketitle
\end{frame}

%\begin{frame}{Plan}
%\tableofcontents
%\end{frame}

\section{Introduction}
\subsection{Histoire}
\begin{frame}{Titre de la frame}

\begin{block}{Titre bloc}
\begin{itemize}
\item Contenu bloc
\end{itemize}
\end{block}

\begin{alertblock}{Bloc 2}
\begin{itemize}
\item Contenu 2
\end{itemize}
\end{alertblock}

\begin{exampleblock}{Bloc 3}
\begin{enumerate}
\item Premièrement
\item Deuxièmement
\end{enumerate}
\end{exampleblock}

\begin{itemize}
\item Un item
\item Deux items
\end{itemize}




\end{frame}



\end{document}
